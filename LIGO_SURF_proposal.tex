\documentclass{article}
\usepackage{amsmath}
\usepackage{graphicx}
\usepackage[margin = 1in]{geometry}

\title{LIGO SURF Proposal 2017}
\author{Shreya Anand \thanks{under the guidance of Dr. Alex Urban}}
\date{May 15, 2017}

\begin{document}
\begin{titlepage}
\maketitle
\end{titlepage}

\section{Introduction}

In the current advanced gravitational wave detectors era, scientists predict there to be a significant increase in the number of detections annually by LIGO and other ground-based detectors, such as Virgo, KAGRA, and LIGO-India, projected to come online within the next five years.  The only existing confirmed detections were of the binary black hole (BBH) systems GW150914 and GW151226 observed at both the LIGO Hanford and LIGO Livingston detector sites.  During the second observing run (O2) currently underway, there is a greater chance of detecting other astrophysical sources of gravitational waves, including neutron-star black hole (NS-BH) and neutron star-neutron star (BNS) systems, due to the improved sensitivity of Advanced LIGO as compared to initial LIGO.  Similarly, with the onset of third generation detectors such as Cosmic Explorer and Einstein Telescope, we anticipate the ability to confirm at least 20-30 NS-BH or BNS systems detections using these advanced detectors.  Currently, with a very limited sample population of compact merger objects identified from gravitational wave observations, making definitive statements about the nature of compact binary systems in general,  is difficult.  However, as the number of detections increases, so does our ability to better understand compact binary systems and their host environments.

In the past two gravitational wave detections in 2015 and 2016 respectively, chirp masses of the binary sources were measured to high precision, which provided strong evidence suggesting that the mass ratios of binary black holes (BBH) can vary widely from system to system.  However, measurements of both sky localization and luminosity distance to the source, both closely coupled with the signal-to-noise ratio (S/N) of the signals, had large associated uncertainties.  Specifically, GW150914 was localized to about 850 ${\rm deg}^{2}$ on the sky, and determined to have a luminosity distance of $410^{+160}_{-180} {\rm Mpc}$, while GW151226 was localized to within 1400 ${\rm deg}^{2}$ at a luminosity distance of $440^{+180}_{-190}$ Mpc [1].  The poor localization constraint for the past two confirmed detections, as well as the rather imprecise distance estimate has hindered our ability to gain insight into the astrophysical sources of the gravitational wave signals.  Therefore, improvement in the uncertainties with which we measure both intrinsic and extrinsic parameters that allow us to characterize a gravitational wave source is key to doing astrophysics with gravitational waves.      
 
Measurements of the intrinsic parameters, including the chirp mass, spin, luminosity distance, and coalescence time, and extrinsic parameters, including amplitude, phase, and arrival time of a gravitational wave signal associated with a compact binary system are made using matched filtering techniques to pair the signal with its most closely fitting template, constructed from numerical relativity [1].  For a correctly matched signal and template, the intrinsic parameters of the astrophysical system are well described by the 'characteristics' of the template.  Extrinsic parameters, however, depend solely on the positions and orientations of the detectors with respect to each other, and to the binary system .  Here, it is important to note that the localization arcs on the sky are a result of the timing triangulation between detectors, constrained by the signal amplitude and phase (and their associated uncertainties) [4].  Both intrinsic and extrinsic parameters can be inferred from the signal itself, which is observed by ground-based detectors as the dimensionless strain

\begin{equation}    y_i(t) = x_i(t;\theta) + n_i(t) \end{equation}

in the time domain, while in frequency space it is given by 

\begin{equation} Y_i(\omega) = X_i(\omega; \theta) + N_i(\omega) \end{equation}

where X, the signal data, and N, the noise, are functions of $\theta$, a parameter that describes the source system, $\omega$, the angular frequency of the signal, and the time. X, the signal data, is given as a function of $\omega$ as:

\begin{equation} X = \rho H(\omega)e^{(i\gamma - i\omega\tau)} \end{equation}

where $\rho$ is the signal-to-noie ratio, $H(\omega)$ describes the portion of the signal that depends on the intrinsic parameters of the system, $\gamma$ is the signal phase, and $\tau$ is the arrival time of the signal to the detector [4].

Both intrinsic and extrinsic parameters have associated uncertainties that are comprised of both systematic as well as statistical components.  The statistical uncertainties in amplitude, phase, and arrival time can be determined using a Fisher Information matrix (related to the inverse of the covariance matrix of uncertainties in the parameters) analysis, assuming the likelihood is distributed as a gaussian [4].  The systematic uncertainty, on the other hand, comes from calibration parameters.  However, the effect of detector calibration on parameter estimation with next generation detectors is as of yet unknown; it is a subject we hope to further explore in our study. Vitale et. al, 2016 uses a Bayesian analysis to address the case of binary black hole systems, specifically, and predict how well intrinsic and extrinsic parameters can be measured for such systems with advanced detector networks.  Bayesian analysis, which relies on the Fisher information, is built on the assumption that given the data and prior information about how the data is distributed (known as the prior), the probability of an event occuring is equal to the product of the prior and the likelihood of the data given the prior, marginalized over a number of different hypotheses [4].  This type of analysis is often used in the context of extracting information from gravitational wave observations. 

As mentioned earlier, observations predicted from next generation detectors include not only BBH sources, but also compact binary systems with at least one, if not two neutron stars.  Such sources are the hypothesized progenitors of short gamma-ray bursts, which are thought to occur immediately following the coalescence of the neutron stars [2].  Detections of inspiraling binary systems with a neutron star involved in the interaction opens up the possibility of coupling light - electromagnetic spectra - with gravitational wave signals to use as a probe for the astrophysics of the source system.  Observing electromagnetic (EM) counterparts to BBH merger events, while possible, is much less probable.  Using the combined information from gravitational wave observations and their electromagnetic counterparts serve three main important functions in studying NS-BH and BNS mergers.  The first is that using the time evolution from a gravitational wave signal, one can study the central engine and bulk motion of the matter in a short GRB, in addition to studying the matter surrounding (external to) the system, using electromagnetic observations [2].  Secondly, the chances of identifying the host galaxy increase immensely when gravitational wave observation is paired with an EM counterpart [4].  By investigating certain properties of the host galaxy environment such as the metallicity and star formation rate, we might be able to better understand how binary systems with neutron stars form.  Third, binary neutron star systems can function as 'standard sirens'; gravitational wave observations provide an independent measurement of luminosity distance to the sources, and the electromagnetic spectra provide a measure of the redshifts to the source [3].  NS-BH coalescences are also potential standard sirens; knowing the specific mass at which the black hole tidally disrupts the neutron star, from gravitational wave parameter estimates we can use the redshifted mass and distance measurements to determine $H_{0}$.  Assuming a flat, $\Lambda$CDM cosmology, the luminosity distance $D_L$ is related to the redshift via the equation:

\begin{equation} D_L = \frac{c(1+z)}{H_0}\int_{0}^{z} \frac{dz'}{[\Omega_M(1+z')^3 + \Omega_{\Lambda}(1+z')^{3(1+\omega)}]^{1/2}} \end{equation}

where $H_0$ is the hubble constant, z is the redshift, c is the speed of light, $\Omega_M$ and $\Omega_{\Lambda}$ are the dimensionless matter density and dark energy density parameters in the universe, and $\omega$ determines the equation of state of dark energy.  Since the two measures described above are calibration-independent, compact binary coalesences can function as standard sirens that have the potential of being used the same way as Type Ia supernovae to measure distance scales (and the expansion rate) of the universe.  

However, observing the EM counterparts to makes such cosmological measurements presents significant challenges.  Optical counterparts to the signals being examined are often faint, and require rapid follow-up, which is not usually practical [2].  Furthermore, the degree of localization acheived using current generation detectors thus far is too poor to seriously hunt for the host galaxies of these coalescence events, and require telescopes with a large field of view [2].  Measuring parameters to greater precision, using next generation detectors is key in order to extract the full potential from the combined use of gravitational and electromagnetic signals when there are a wealth of NS-BH and NS-NS binary detections. Thereby, we intend to extend the study performed in Vitale et. al 2016 by performing a similar parameter estimation, but using sample populations of BNS and NS-BH rather than BBH systems.  We hope to not only provide interesting predictions concerning the correlation of telescope observations of BNS systems with gravitational wave observations, but also to test whether gravitational waves can be used as a way of determining whether BNS mergers can account for all the short gamma-ray bursts observed.

If, as we expect, there is an increased probability of observing EM signatures associated with BNS mergers, we can constrain the hubble constant in the local universe by exploiting the 'standard siren' property of the merger systems.  This procedure for constraining H0, as well as other cosmological parameters, is detailed in the Sathyaprakash, Schutz, and Van Den Broeck paper on cosmography with third generation detector networks (2009).  As another aspect of our project, we propose to follow the methods outlined in Sathyaprakash et al. 2009 to provide a measurement of $H_0$ in our local universe, using the simulated population of binary neutron stars we generate, taking into account the predicted effect of calibration uncertainty on this measurement (which we aim to further investigate into).

In the following paragraphs, we provide a sketch for our proposed summer research project.  First, we will discuss what is already known about next generation detector networks.  Next, we give a succinct summary of the telescopes we plan to use.  We will then provide a brief overview of the relevant background on short GRBs as a possible outcome of a NS-BH or BNS merger, and describe how we plan to use short GRBs within the scope of our research.  Our description of short GRBs will also involve how to use them as a calibration independent measure of the Hubble constant. The bulk of the remainder of this proposal details our projected approach to determining how well we will be able to do parameter estimation, and thereby electromagnetic follow-up, with BNS and NS-BH merger signals detected by next generation gravitational wave detectors.

\section{Project Outline}

In order to probe astrophysical systems well enough to do astronomy with them, next generation detector networks are required.  Gravitational wave detectors that have already been built, or are currently under construction, are considered second generation detectors, while third generation detectors refer to gravitational wave detectors that will have enhanced sensitivity as a result of improved design and technological advancement.  Among the third generation detectors being planned currently are $A+$, Einstein Telescope (ET-D), and Cosmic Explorer (CE).  The first of the three detectors is the result of a modification of the Advanced LIGO detectors to include better coatings, quantum light squeezing, and heavier test masses [5].  Einstein Telescope, located in Europe, is a proposal for an underground detector, with 10 km arms connecting three interferometers in a triangle to better distinguish GW polarizations from other noise, and for increased sensitivty [5].  Cosmic Explorer, the most ambitious project of the three, will be an L-shaped detector with 40 km arms designed to minimize noise sources besides gravitational waves [5].    
 
While sensitive third generation detectors will greatly enrich our ability to do astrophysics with gravitational waves, matching the signals to their EM counterparts enable scientists to use the combined spectral and gravitational wave data to understand more about the source system's formation and evolution.  Sensitive telescopes that are optimal for observing NS-NS and NS-BH binaries are necessary in order to do EM follow-up concurrently with gravitational wave detections.  Over the course of our project, we plan on using the LSST Camera, Zwicky Transient Factory, and Pan-Starrs in order to predict what kinds of gravitational wave-emitting systems are more likely to be observable by optical facilities.  The sensitivities of these telescopes range from ~ 20-25 magnitudes.  Among the three telescopes, Zwicky Transient Facility (ZTF) has the largest field-of-view of 47 $deg^2$, but also the longest slew rate (15 s).  The other two instruments have comparable FOVs; the slew rates between the two differ by five seconds.  

More specifically than the observability of EM counterparts in general, we are interested in the observability of short hard gamma-ray bursts, both by gravitational wave and optical observing facilities.  Though the formation mechanism of their BNS and NS-BH progenitors is still under investigation, theories predict their formation via 1) association of neutron stars in isolated systems 2) gravitational interaction of neutron stars within a dense cluster, and 3) accretion induced collapse of binary systems of white dwarfs.  If NS-NS and NS-BH systems are, indeed the progenitors of short hard gamma ray bursts, the likelihood of a coincident detection of a gravitational wave signal and an electromagnetic counterpart is increased due to the beaming of the resulting jet, and the EM afterglow.

Thus we intend to simulate a population of NS-NS and NS-BH binary systems based on realizations of a catalog of these binaries, as well as short hard gamma ray bursts, from ZTF, Pan-Starrs, and LSST.  With next generation detectors specifically, we expect the EM counterparts to be dominated by GRBs and afterglows, due to the ability of next generation detectors to detect higher redshift sources, and that previous research has suggested that afterglows will have a beaming angle of about 40-50 degrees.  We intend to use these estimates to model the short GRB afterglows.  However, amongst a population of singly or doubly degenerate compact binaries, we only expect a fraction of them to be visible to telescopes, since kilonovae (optical EM counterparts) give weak signals, and the beaming orientation relative to telescopes may result in a significant portion of short GRBs being missed.  We will assume, therefore, based on Abbott et al. 2016 that the upper limit on the number of BNS mergers is 3600 $yr^{-1}Gpc^{-3}$, and, taking the optimistic fraction that 1 - cos(50) of the bursts are detected, we find that 36 percent, or 1000 of such BNS mergers, and 500-1000 of the NS-BH mergers will produce detectable EM counterparts per year per gigaparsec.  In reality the number we expect to detect will be less than that, due to the sky coverage of our telescopes and LIGO's duty cycle.  In simulating the population of NS-NS and NS-BH binaries, we will assume a prior probability distribution that the sources are uniformly distributed in comoving volume, which makes the least assumption about our sources.  This assumption yields a redshift distribution for our sources that we can pair with our measured luminosity distances (assuming a standard, FRW cosmology) from the gravitational waveforms we simulate for the NS-NS and NS-BH sources.  Using a similar methodology as outlined in Van de Broeck et al., 2009, but with a network of next generation detectors, we seek to determine the Hubble constant in the local universe using the redshifts and distance measures we obtain.  Measuring $H_{0}$ to a higher precision in the local universe is not as interesting as the approach taken to measure $H_{0}$; what is novel about this technique is the ability to verify the precise cosmological measurements already made via independent means, and thereby strengthen or refute previous cosmological claims.

Simulating the gravitational wave signals for BNS and NS-BH sources with next generation detectors remains an active area of research.  Since a second component of our project involves extending the binary black hole study performed by Vitale and Evans 2016 to include compact binary mergers with neutron stars, we will follow their example to generate signals and make measurements of the parameters we are interested in, namely the intrinsic mass, sky position, spin, and distance to the source.  We share the assumptions made by Vitale and Evans as far as they are universal to both their binary black hole study as well as our study including systems with neutron stars.  In particular, our assumption that the redshifts of the sources from zero to 20, are uniformly distributed in comoving volume is consistent with the assumption made in Vitale and Evans 2016, as well as that the merger rate is not a strong function of the redshift.  We also will consider sources with spins in the range [0,0.98], and only keep the source if it has a signal-to-noise ratio in the range [10,600].  However, our chirp mass distribution will be different because our systems include neutron stars, and we will include neither the black hole-black hole merger specific characteristics nor the same precessing approximant they used in our analysis.  We do though, use the signal-to-noise ratio parameter to determine the detectability of the source.

In repeating the parameter estimation study, our goal is to be able to determine intrinsic mass, sky position, spin, and luminosity distance for our simulated sources, as well as the uncertainties with which we can measure these parameters.  The part of our project related to cosmography should provide an understanding of what degree of precision in measurements is needed to do meaningful cosmological measurements.  If our project is successful, we hope to be able to use both our work on cosmography and parameter estimation to inform scientific requirements for next generation detector networks.

Because our work is a collaborative effort, drawing on previous literature as well as other parallely occuring research, we plan on harnessing the wealth of scientific knowledge that is dispersed among several LIGO research groups on the Caltech campus.  In order to inform our signal modeling, we hope to consult both the Calibration and Instrumentation LIGO working  groups and incorporate their knowledge about the way calibration affects next generation detectors into our signal model.  We also hope to validate our assumptions about the detectability of short GRBs by consulting astronomers working on EM follow-up of gravitational wave signals.  In doing so, our intention is to be able to do the most informed analysis possible to determine scientific requirements for third generation detectors.   
\section{Projected Timeline}
In this section we provide a sketch in the form of a table, detailing our goals and activities for our project over the SURF program dates of June 20 - August 25. \\

\begin{table} [ht]
\caption{Schedule for SURF project}
\begin{tabular}{ p{3cm} p{10cm} }
 \hline
\centering Timeline & Goals and Activities \\ [0.5ex] 
 \hline
 \hline
June 20 - June 27 & introduction to LIGO, tutorials on astrophysics, instrumentation, and noise \\
June 27 - July 8 & develop a concrete plan for project, begin designing experiment \\
July 9 - 14 & attend Amaldi conference for Gravitational Waves \\
mid July & 5 minute update on project - have designed simulation \\
July 17 - 19 & 3 day trip to LIGO Livingston Observatory \\
July 20 - August 18 & perform data analysis, work SURF report \\
August 19 - 24 & finalize SURF report and relevant analysis results \\  
August 25 & LIGO SURF presentations \\
 \hline
\end{tabular}
\end{table}

\section{References}

1.  Abbott, B.P. et al. (LIGO Scientific Collaboration) 2016, Phys. Rev. Lett. 116, 061102 \\
2.  Metzger, B. 2016, arXiv: 1610.09381v1, astro-ph.HE \\
3.  Sathyaprakash B.S., Schutz B.F. and C. V. Den Broeck 2009, Class. Quant. Grav. 27, 216006 \\
4.  Singer, L.P., and Price, L.R. 2016, Phys. Rev. D 93, 024013 \\
5.  Vitale, S. and Evans, M. 2016, Phys. Rev. D 95, 064052 \\
  
\end{document}
