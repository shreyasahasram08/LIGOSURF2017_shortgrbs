\documentclass{article}
\usepackage{amsmath}
\usepackage{graphicx}
\graphicspath{ {} }

\begin{document}
\section{Introduction}

In the current advanced gravitational wave detectors era, scientists predict there to be a significant increase in the number of detections annually by LIGO and other ground-based detectors, such as Virgo, KAGRA, and LIGO-India, projected to come online within the next five years.  The only existing confirmed detections were of the binary black hole systems GW150914 and GW151226 observed at both the LIGO Hanford and LIGO Livingston detector sites.  During the second observing run (O2) currently underway, there is a greater chance of detecting other astrophysical sources of gravitational waves, including neutron-star black hole and neutron star-neutron star systems, due to the improved sensitivity of Advanced LIGO as compared to initial LIGO.  Similarly, with the onset of third generation detectors such as Cosmic Explorer and Einstein Telescope, we anticipate the ability to confirm at least 20-30 neutron star-black hole or neutron star-neutron star systems detections using these advanced detectors.  Currently, with a very limited sample population of compact merger objects identified from gravitational wave observations, making definitive statements about the nature of compact binary systems in general,  is difficult.  However, as the number of detections increases, so does our ability to better understand compact binary systems and their host environments.

In the past two gravitational wave detections in 2015 and 2016 respectively, chirp masses of the binary sources were measured to high precision, which provided strong evidence suggesting that the mass ratios of binary black holes can vary widely from system to system.  However, measurements of both sky localization and luminosity distance to the source, both closely coupled with the signal-to-noise ratio (S/N) of the signals, had large associated uncertainties.  Specifically, GW150914 was localized to about 600 $deg^2$ on the sky, and determined to have a luminosity distance of $410^{+160}_{-180} Mpc$, while GW151226 was localized to within 1400 $deg^{2}$ at a luminosity distance of $440^{+180}_{-190}$ Mpc (Abbott et al., 2016).  The poor localization constraint for the past two confirmed detections, as well as the rather imprecise distance estimate has hindered our ability to gain insight into the astrophysical sources of the gravitational wave signals.  Therefore, improvement in the uncertainties with which we measure both intrinsic and extrinsic parameters that allow us to characterize a gravitational wave source is key to doing astrophysics with gravitational waves.      
 
Measurements of the intrinsic parameters, including the chirp mass, spin, luminosity distance, and coalescence time, and extrinsic parameters, including amplitude, phase, and arrival time of a gravitational wave signal associated with a compact binary system are made using matched filtering techniques to pair the signal with its most closely fitting template, constructed from numerical relativity.  For a correctly matched signal and template, the intrinsic parameters of the astrophysical system are well described by the 'characteristics' of the template.  Extrinsic parameters, however, depend solely on the positions and orientations of the detectors with respect to each other, and to the binary system.  Here, it is important to note that the localization arcs on the sky are a result of the timing triangulation between detectors, constrained by the signal amplitude and phase (and their associated uncertainties).  Both intrinsic and extrinsic parameters can be inferred from the signal itself, which is observed by ground-based detectors as the dimensionless strain

\begin{equation}    y_i(t) = x_i(t;\theta) + n_i(t) \end{equation}

in the time domain, while in frequency space it is given by 

\begin{equation} Y_i(\omega) = X_i(\omega; \theta) + N_i(\omega) \end{equation}

where X, the signal data, and N, the noise, are functions of $\theta$, a parameter that describes the source system, $\omega$, the angular frequency of the signal, and the time. X, the signal data, is given as a function of $\omega$ as:

\begin{equation} X = \rho H(\omega)e^{(i\gamma - i\omega\tau)} \end{equation}

where $\rho$ is the signal-to-noie ratio, $H(\omega)$ describes the portion of the signal that depends on the intrinsic parameters of the system, $\gamma$ is the signal phase, and $\tau$ is the arrival time of the signal to the detector.

Both intrinsic and extrinsic parameters have associated uncertainties that are comprised of both systematic as well as statistical components.  The statistical uncertainties in amplitude, phase, and arrival time can be determined using a Fisher Information matrix (related to the inverse of the covariance matrix of uncertainties in the parameters) analysis, assuming the likelihood is distributed as a gaussian.  The systematic uncertainty, on the other hand, comes from calibration parameters.  However, the effect of detector calibration on parameter estimation with next generation detectors is as of yet unknown; it is a subject we hope to further explore in our study. Salvatore et. al, 2016 uses a Bayesian analysis to address the case of binary black hole systems, specifically, and predict how well intrinsic and extrinsic parameters can be measured for such systems with advanced detector networks.  Bayesian analysis, which relies on the Fisher information, is built on the assumption that given the data and prior information about how the data is distributed (known as the prior), the probability of an event occuring is equal to the product of the prior and the likelihood of the data given the prior, marginalized over a number of different hypotheses.  This type of analysis is often used in the context of extracting information from gravitational wave observations. 

As mentioned earlier, observations predicted from next generation detectors include not only binary black hole sources, but also compact binary systems with at least one, if not two neutron stars.  Such sources are the hypothesized progenitors of short gamma-ray bursts, which are thought to occur immediately following the coalescence of the neutron stars.  Detections of inspiraling binary systems with a neutron star involved in the interaction opens up the possibility of coupling light - electromagnetic spectra - with gravitational wave signals to use as a probe for the astrophysics of the source system.  Observing electromagnetic (EM) counterparts to black hole-black hole merger events, while possible, is much less probable.  Using the combined information from gravitational wave observations and their electromagnetic counterparts serve three main important functions in studying neutron star-black hole and neutron star-neutron star mergers.  The first is that using the time evolution from a gravitational wave signal, one can study the central engine and bulk motion of the matter in a short GRB, in addition to studying the matter surrounding (external to) the system, using electromagnetic observations.  Secondly, the chances of identifying the host galaxy increase immensely when gravitational wave observation is paired with an EM counterpart.  By investigating certain properties of the host galaxy environment such as the metallicity and star formation rate, we might be able to better understand how binary systems with neutron stars form.  Third, binary neutron star systems can function as 'standard sirens'; gravitational wave observations provide an independent measurement of luminosity distance to the sources, and the electromagnetic spectra provide a measure of the redshifts to the source.  Assuming a flat, $\Lambda$CDM cosmology, the luminosity distance $D_L$ is related to the redshift via the equation:

\begin{equation} D_L = \frac{c(1+z)}{H_0}\int_{0}^{z} \frac{dz'}{(\Omega_M(1+z')^3 + \Omega_{\lambda}(1+z'))^{3(1+\Omega)^{1/2}}} \end{equation}

where $H_0$ is the hubble constant, z is the redshift, c is the speed of light, $\Omega_M$ and $\Omega_{\lambda}$ are the dimensionless matter density and dark energy density parameters in the universe, and $\omega$ determines the equation of state of dark energy.  Since these two measures are calibration-independent, compact binary coalesences can function as standard sirens that have the potential of being used the same way as Type Ia supernovae to measure distance scales (and the expansion rate) of the universe.  

Observing EM counterparts to gravitational wave signals presents significant challenges.  Optical counterparts to the signals being examined are often faint, and require rapid follow-up, which is not usually practical.  Furthermore, the degree of localization acheived using current generation detectors thus far is too poor to seriously hunt for the host galaxies of these coalescence events, and require telescopes with a large field of view.  Measuring parameters to greater precision, using next generation detectors is key in order to extract the full potential from the combined use of gravitational and electromagnetic signals when there are a wealth of neutron star-black hole and neutron star-neutron star binary detections. Thereby, we intend to extend the study performed in Salvatore et. al 2016 by performing a similar parameter estimation, but using a sample population of binary neutron stars, rather than binary black holes.  We hope to not only provide interesting predictions concerning the correlation of telescope observations of binary neutron star systems with gravitational wave observations, but also to test whether gravitational waves can be used as a way of determining whether binary neutron star mergers can, in fact be the sources of short gamma ray bursts.

If, as we expect, there is an increased probability of observing EM signatures associated with binary neutron star mergers, we can constrain the hubble constant in the local universe by exploiting the 'standard siren' property of the merger systems.  This procedure for constraining H0, as well as other cosmological parameters, is detailed in the Sathyaprakash, Schutz, and Van Den Broeck paper on cosmography with third generation detector networks (2009).  As another aspect of our project, we propose to follow the methods outlined in Sathyaprakash et al. 2009 to provide a measurement of H0 in our local universe, using the simulated population of binary neutron stars we generate, taking into account the predicted effect of calibration uncertainty on this measurement (which we aim to further investigate into).

\end{document}
